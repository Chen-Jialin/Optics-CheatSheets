%Optics Cheat Paper for Mid
\documentclass[10pt,a4paper]{article}
\linespread{1}
\usepackage[UTF8]{ctex}
\usepackage{bm}
\usepackage{amsmath}
\usepackage{amssymb}
\usepackage{geometry}
\geometry{top=0cm,bottom=3cm,left=0cm,right=0cm}
\title{Optics Cheat Paper for Mid}
\author{陈稼霖 \and 45875852}
\date{2018.11.21}
\begin{document}\scriptsize
\textbf{Chap1}

\textbf{斯涅耳定律}$n_1\sin i_1=n_2\sin i_2$,光密$\to$光疏时\textbf{全反射角}$i_c=\arcsin\frac{n_2}{n_1}, n_2<n_1$

\textbf{光纤}中光传播规律$\frac{d^2r}{dz^2}=\frac{n}{n_0^2\cos^2\theta_0}\frac{dn}{dr}=\frac{1}{2n_0^2\cos^2\theta_0}\frac{dn^2}{dr}$,其中$r$--到光纤轴的距离,$z$--沿轴传播的距离,$\theta$--与轴夹角,带有下标$0$ 为初始条件

\textbf{三棱镜色散最小偏向角}条件$n=\frac{\sin\frac{\alpha+\delta_{\min}}{2}}{\sin\frac{\alpha}{2}}$,其中$\alpha$--顶角

\textbf{辐射能通量}$\Psi=\int_0^{+\infty}\psi(\lambda)d\lambda$;\textbf{视见函数}$V(\lambda)=\frac{\Psi_{555}}{\Psi_\lambda}$,\textbf{光通量}$\Phi=K_{\max}\int_0^{+\infty}V(\lambda)\psi(\lambda)d\lambda$,单位流明$lm$,其中$K_{\max}=683lm/W$

\textbf{发光强度}:沿某一方向单位立体角内发出的光通量,对点光源以$\bm{r}$ 为轴的立体角元$d\Omega$,$I=\frac{d\Phi}{d\Omega}$,单位坎德拉$cd=lm/sr$

\textbf{亮度}:沿某一方向单位投影面积的发光强度,对面光源上面元$dS$沿$\bm{r}$方向立体角$\Omega$,$B=\frac{dI}{dS^*}=\frac{dI}{dS\cos\theta}$,单位$lm/(m^2\cdot sr)=10^{-4}lm/(cm^2\cdot sr)=10^{-4}\text{熙提}sb$,其中$\theta$-- 面元法向量与$\bm{r}$ 夹角

\textbf{朗伯发光体}:$I\propto\cos\theta\Longrightarrow B=constant$

\textbf{照度}:照射在单位面积上的光通量$E=\frac{d\Phi'}{dS'}$,单位勒克斯$lx=lm/m^2$或辐透$ph=lm/cm^2$,将$\Phi$换为$\Psi$则得\textbf{辐射照度}

发光强度为$I$的\textbf{点光源}照到距离为$r$的$dS'$上,$E=\frac{I\cos\theta'}{r^2}$

\textbf{面光源}$dS$(法线$n$)照到距离为$r$的$dS'$(法线$n'$)上,$E=\iint_{\text{光源表面}S}\frac{BdS\cos\theta\cos\theta'}{r^2}$

\textbf{Chap2}

\textbf{虚光程}:虚物(像)到透镜的光程$=-$物(像)方折射率$\times|$虚物(像)到透镜的距离$|$

折射球面\textbf{齐明点}$\overline{QC}=\frac{n'}{n}r,\overline{Q'C}=\frac{n}{n'}r$,其中$C$球心,$C,Q,Q'$共线,做球面上任一点$M$,延长$QC$交球面于$A$,则$\frac{\sin\angle MQC}{\sin\angle MQ'C}=\frac{\overline{AQ'}}{\overline{QA}}$

傍轴光线在\textbf{单球面上的折射}$\frac{n'}{s'}+\frac{n}{s}=\frac{n'-n}{r}$,$s$--物距,$s'$--像距,$n$--物方折射率,$n'$--像方折射率,$r$--球面曲率半径,物、像方焦距$f=\frac{nr}{n'-n},f'=\frac{n'}{n'-n}$,从而$\frac{f'}{s'}+\frac{f}{s}=1$,法则(入射光左$\to$右):
\begin{itemize}
  \item[I] 物在顶点之左(实物),则$s>0$
  \item[II] 像在顶点之右(实像),则$s'<0$
  \item[II'] (反射)像在顶点之左(实像),则$s'>0$,$\frac{1}{s'}+\frac{1}{s}=-\frac{2}{r}$或$f=-\frac{r}{2}$
  \item[III] 球心在顶点之右(凸球面),则$r>0$
\end{itemize}

\textbf{横向放大率}$V=\frac{y'}{y}=-\frac{ns'}{n's}$,其中$y$--物高,$y'$--像高;(反射)$V=-\frac{s'}{s}$
\begin{itemize}
  \item[IV] 像在光轴上方,$y>0$
\end{itemize}

\textbf{拉格朗日-亥姆霍兹定理}:$ynu=y'n'u'$,其中$u$-光线对于光轴的倾角
\begin{itemize}
  \item[V] 从光轴逆时针转到光线为小角时,$u>0$
\end{itemize}

\textbf{薄透镜成像(高斯形式)}$\frac{f'}{s}+\frac{f}{s}=1$,若$f=f'$,则$\frac{1}{s'}+\frac{1}{s}=\frac{1}{f}$;其中焦距$f=\frac{n}{\frac{n_L-n}{r_1}+\frac{n'-n_L}{r_2}},f'=\frac{n'}{\frac{n_L-n}{r_1}+\frac{n'-n_L}{r_2}}$,其中$n_L$--透镜折射率,$r_1,r_2$--物、像方曲率半径,当$n=n'=1$,磨镜者公式$f=f'=\frac{1}{(n_L-1)(\frac{1}{r_1}-\frac{1}{r_2})}$

\textbf{薄透镜成像(牛顿形式)}$xx'=ff'$,其中$x,x'$--从焦点算起的物、像距
\begin{itemize}
  \item[VI] 当物在$F$之左,则$x>0$
  \item[VII] 当像在$F'$之右,则$x'>0$
\end{itemize}

\textbf{横向放大率}$V=-\frac{ns'}{n's}=-\frac{fs'}{f's}=-\frac{f}{x}=-\frac{x'}{f'}$

\textbf{密接透镜组光焦度}直接相加$P=P_1+P_2$,其中$P_{(1)/(2)}=\frac{1}{f_{(1)(2)}}$,单位屈光度$D=m^{-1}$,眼镜度数$=$屈光度$\times100$

\textbf{作图法}:
\begin{itemize}
  \item[1] 当$n=n'$,通过光心$O$光线方向不变
  \item[2] 通过$F(F')$光线,经透镜后(前)平行于光轴
  \item[3] 通过物(像)方焦面上一点$P$的光线,经透镜前(后)和$OP$平行
\end{itemize}

\textbf{主点和主面}:$V=1$,对薄透镜,物、像方主点(面)重合,作图时主面之间光线一律平行光轴
\begin{itemize}
  \item[I'] 物或$F$在$H$之左,则$s(f)>0$
  \item[II'] 像或$F'$在$H$之右,则$s'(f')>0$
\end{itemize}

\textbf{角放大率}$W=\frac{\tan u'}{\tan u}=-\frac{s}{s'}$;$VW=\frac{f}{f'}$;亥姆霍兹公式$yn\tan u=y'n'\tan u'$,拉格朗日-亥姆霍兹定理在非傍轴条件下的推广

\textbf{理想光具组联合}$f=-\frac{f_1f_2}{\Delta},f'=-\frac{f_1'f_2'}{\Delta},X_H=f_1\frac{\Delta+f_1'+f_2}{\Delta}=f_1\frac{d}{\Delta},X_{H'}=f_2'\frac{\Delta+f_1'+f_2}{\Delta}=f_2'\frac{d}{\Delta}$,其中$\Delta$--前一透镜像方焦点和后一透镜物方焦点间距,$d$--两透镜间距
\begin{itemize}
  \item[VII] $F_2$在$F_1'$之右,则$\Delta>0$
  \item[IX] $H_2$在$H_1'$之右,则$d>0$
  \item[X] $H$在$H_1$之左,则$X_H>0$;$H'$在$H_2'$之右,则$X_{H'}>0$
\end{itemize}

\textbf{照相机}:$s\to+\infty,s'\approx f'$,光阑越小,则曝光时间越长,景深越大;$\frac{\delta x'}{\delta x}=-\frac{f^2}{x^2}$,从而给定$f$,$x$越小,景深越小

\textbf{眼睛}:通过调节晶状体曲率实现调焦成像,物、像方焦距不等;睫状肌完全松弛和最紧张时清晰成像的点--远点和近点;成像大小与物的视角$w$成正比;最舒适的物距--明视距离$s_0=25cm$

\textbf{放大镜和目镜}:焦距很小;物体视角最大值$w=\frac{y}{s_0}$;放大镜紧贴眼镜,物体放在放大镜焦点内一个小范围内,$0\geq x\geq-\frac{f^2}{s_0+f}$,$\because f\ll s_0$,$\therefore|x|\ll f$,物对光心所张视角即为像对眼所张视角$w'=\frac{y}{f}$;视角放大率$M=\frac{w'}{w}=\frac{s_0}{f}$

\textbf{显微镜}:物镜焦距很小,目镜即为放大镜;视角放大率$M=\frac{w'}{w}=\frac{y_1/f_E}{y/s_0}=\frac{y_1}{y}\frac{s_0}{f_E}=V_OM_E$,其中$y_1$--物镜成像高,$f_E$ 目镜焦距,$V_0=\frac{y_1}{y}$--物镜横向放大率,$M_E=\frac{s_0}{f_E}$--目镜视角放大率;$V_O=-\frac{\Delta}{f_O}$,其中$\Delta$--$F_O'$ 与$F_E$间距即光学筒长,$f_O$--物镜焦距,从而$M=-\frac{s_0}{f_O}\frac{\Delta}{f_E}$

\textbf{望远镜}:物镜焦距较大,$F_O'$和$F_E$几乎重合;视角放大率$M=\frac{w'}{w}=\frac{y_1/f_E}{-y_1/f_O}=-\frac{f_O}{f_E}$(倒像)

\textbf{棱镜光谱仪}:点光源发射光线先经过准直管成为平行光线,被三棱镜色散后相同波长的光线仍为平行光,经过望远物镜汇聚成一点;焦色散本领$D=\frac{d\delta}{d\lambda}$;由于三棱镜厚度,谱线弯曲,在最小偏向角条件下,弯曲最小,此时$D=\frac{d\delta_{min}}{d\lambda}=\frac{d\delta_{min}}{dn}\frac{dn}{d\lambda}=(\frac{dn}{d\delta_{min}})^{-1}\frac{dn}{d\lambda}=\frac{2\sin\frac{\alpha}{2}}{\cos\frac{\alpha+\delta_{min}}{2}}\frac{dn}{d\lambda}$

\textbf{(对轴上物点)孔径光阑}--对光束孔径限制最多的光阑,\textbf{入(出)射孔径角}$u_0,u_0'$--被孔径光阑限制的边缘光线与物(像)方光轴夹角,\textbf{入(出)射光瞳}-- 孔径光阑在物(像)方的共轭;对于不同的共轭点,可以有不同的孔径光阑和光瞳

\textbf{(对于轴外物点)主光线}--通过入射光瞳中心$O$(自然通过出射光瞳中心$O'$)的光线;随着主光线倾角增大,最终其将被某光阑所先限制,该光阑即\textbf{视场光阑};入(出)射视场角--物(像)方主光线$PO,O'P'$和光轴的倾角$w_0,w_0'$;视场--物平面上被$w_0$限制的范围;视场之外物点也可成暗像;\textbf{渐晕}--像平面内视场边缘逐渐昏暗,若要消除渐晕,可以使视场光阑与物平面重合;显微镜和望远镜的视场光阑与中间像重合,其相对于目镜的共轭位于像平面上;入(出)射窗--视场光阑在物(像)方的共轭

\begin{itemize}
  \item \textbf{(单色差)球差}--光轴上一物点发出的光线经球面折射后不再交于一点,高度为$h$的光线焦点与傍轴光线焦点间距$\delta s_h$,凸透镜在左为负;可通过配曲法即调节$\frac{r_1}{r_2}$或复合(凸凹)透镜消除某高度的球差
  \item \textbf{彗差}--通过光瞳不同同心圆环的的光线所成像半径和圆心均不同,形成彗星状的光斑;可通过配曲法或复合透镜来消除
  \item (阿贝正弦条件--消除球差前提下傍轴物点以大孔径光束成像的充要条件$ny\sin u=n'y'\sin u'$)\textbf{像散}--当物点远离光轴,出射光束的截面成椭圆,在子午焦线和弧矢焦线两处退化为互相垂直的直线,称散焦线,在两散焦线之间某处截面呈圆形,称最小模糊圆,该处光束最汇聚;通过复杂透镜组消除
  \item \textbf{像场弯曲}--散焦线和最小模糊圆的轨迹为一曲面;通过在透镜前适当位置放置一光阑矫正
  \item \textbf{畸变}--各处放大率不同;远光轴处放大率偏大则枕形畸变,反之桶形畸变;畸变种类与孔径光阑位置有关,光阑置于凸透镜前则桶形,后则枕形
  \item \textbf{(色差)位置(轴向)/放大率(横向)色差}--由于不同波长光折射率不同导致焦距/放大率不同;消色差胶合透镜$P_1=(n_1-1)K_1,P=P_1+P_2=(n_1-1)K_1+(n_2-1)K_2,P_C=(n_1C-1)K_1+(n_{2C}-1)K_2,\bm{P_F-P_C=(n_{1F}-n_{1C})K_1+(n_{2F}-n_{2C})K_2=0}$,其中$K_1=\frac{1}{r_1}-\frac{1}{r_2},K_2=\frac{1}{r_2}-\frac{1}{r_3}$,对于非黏合的透镜组,$P=P_1+P_2-P_1P_2d=(n_1-1)(K_1+K_2)-(n-1)^2K_1K_2d,\frac{dP}{dn}=K_1+K_2-2(n-1)K_1K_2d=0,\bm{d=\frac{K_1+K_2}{2(n-1)K_1K_2d}=\frac{P_1+P_2}{2P_1P_2}=\frac{f_1+f_2}{2}}$
\end{itemize}

$B=\frac{E}{\pi}$,$E$--屏上像的照度,$B$--观察者看到像的亮度

\textbf{像的亮度}$\Phi=\int B\sigma\cos u\Omega=\int_0^{u_0}B\sigma\cos u\sin udu\int_0^{2\pi}d\varphi$,对朗伯体,$\Phi=\pi B\sigma\sin^2u_0,\frac{B'}{B}=\frac{\Phi'\sin^2u_0\sigma}{\Phi\sin^2u_0'\sigma'}$,透光系数$k=\frac{\Phi'}{\Phi}\leq1$,利用正弦条件$ny\sin u_0=n'y'\sin u_0'$和$\frac{\sigma}{\sigma'}=\frac{y^2}{y'^2}$,故$\frac{B'}{B}=k(\frac{n'}{n})^2$,其中$\sigma$--物的面积,$\sigma'$--像的面积

\textbf{像的照度}$E=\frac{\Phi'}{\sigma'}=\pi B'\sin^2u_0'=k\pi B(\frac{n'}{n})^2\sin^2u_0'=k\pi B\frac{\sin^2u_0}{V^2}\approx k\pi B\frac{u_0^2}{V^2}$

\textbf{拓展光源天然主观亮度}$H_0\triangleq B=(\frac{n'}{n})^2\frac{k\pi B}{4}(\frac{D_e}{f})^2$,其中$D_e$--瞳孔直径,$f$--眼睛焦距

\textbf{Chap3}

\textbf{光波}--横波用标量波处理$U(P,t)=A(P)\cos[\omega t-\varphi(P)]\Leftrightarrow\widetilde{U}=A(P)e^{\pm i[\omega t-\varphi(P)]}=\widetilde{U}(P)e^{-i\omega t}\Leftrightarrow U(P,t)=A(P)\cos[\omega t-\varphi(P)]$,为方便选$-$;\textbf{光强}$I(P)=[A(P)]^2=\widetilde{U}*(P)\widetilde{U}(P)$

\textbf{光波叠加}$U(P,t)=U_1(P,t)+U_2(P,t)$,对同频率$\widetilde{U}(P,t)=\widetilde{U}_1(P,t)+\widetilde{U}_2(P,t)$;强度$I(P)=[A_1(P)]^2+[A_2(P)]^2+2\sqrt{I_1(P)I_2(P)}\cos\delta(P)$,其中$P$点相位差$\delta(P)=\varphi_1(P)-\varphi_2(P)$;若振源等强$I(P)=2A^2[1+\cos\delta(P)]=4A^2\cos^2\frac{\delta(P)}{2}$,其中$\delta(P)=\varphi_1(P)-\varphi_2(P)=\varphi_{10}-\varphi_{20}+\frac{2\pi(r_1-r_2)}{\lambda}$;若振源同相$\delta(P)=\frac{2\pi(r_1-r_2)}{\lambda}$;波程差$\Delta L=r_1-r_2$

\textbf{杨氏双缝干涉条纹间隔}$\Delta x=\frac{\lambda D}{d}$;\textbf{涅菲尔双镜}$\Delta x=\frac{\lambda(B+C)}{2\alpha B}$,其中$B$--光源与双镜交线距离,$C$ -- 双镜交线与屏幕距离,$\alpha$--双镜夹角;\textbf{涅菲尔双棱镜}$\Delta x=\frac{\lambda(B+C)}{2(n-1)\alpha B}$,其中$B$--光源到棱镜距离,$C$--棱镜到屏幕距离,$n$--棱镜折射率,$\alpha$--棱镜底角;\textbf{劳埃德镜}$\Delta x=\frac{\lambda D}{2a}$,其中$D$-- 光源到屏幕距离,$a$-- 光源到镜面距离;\textbf{条纹位移与点光源位移关系}$\delta x=-\frac{D}{R}\delta s$

\textbf{干涉条纹衬比度}$\gamma=\frac{I_{max}-I_{min}}{I_{max}+I_{min}}$;\textbf{对杨氏干涉}$I(x)\sim1+\cos\delta(x)=1+\cos(\frac{2\pi d}{\lambda D}x)=1+\cos(\frac{2\pi x}{\Delta x})$,若光源移动$\delta s$则$I(x)\sim1+\cos[\frac{2\pi d}{\lambda D}(x-\delta x)]=1+\cos[\frac{2\pi d}{\lambda D}(x+\frac{D}{R}\delta s)]=1+\cos\frac{2\pi x}{\Delta x}\cos(\frac{2\pi d}{\lambda R}\delta s)-\sin\frac{2\pi x}{\Delta x}\sin(\frac{2\pi d}{\lambda R}\delta s)$;\textbf{对宽度为$b$的光源}$I(x)=\frac{I_0}{b}\int_{-\frac{b}{2}}^{\frac{b}{2}}[1+\cos\frac{2\pi x}{\Delta x}\cos(\frac{2\pi d}{\lambda R}\delta s)-\sin\frac{2\pi x}{\Delta x}\sin(\frac{2\pi d}{\lambda R}\delta s)]d(\delta s)=I_0[1+\frac{\sin(\pi db/\lambda R)}{\pi db/\lambda R}\cos\frac{2\pi x}{\Delta x}]$,从而$I_{max}=1+|\frac{\sin u}{u}|,I_{min}=1-|\frac{\sin u}{u}|$,衬比度$\gamma=|\frac{\sin u}{u}|$,其中$u=\frac{\pi db}{\lambda R}$


\textbf{杨氏实验光源极限宽度}$b_0\approx\frac{R}{d}\lambda$;光场中\textbf{相干范围的横向线度}$d\approx\frac{R\lambda}{b}=\frac{\lambda}{\varphi}$,其中$\varphi$-- 光源宽度对缝张角;\textbf{相干范围孔径角}$\Delta\theta_0\triangleq\frac{d}{R}\approx\frac{\lambda}{b}$

\textbf{薄膜等厚干涉光程差}$\Delta L=\frac2nh\cos i$,其中$n$--膜折射率,$h$-- 膜厚度,$i$--膜表面折射角,亮纹$\Delta L=k\lambda$,暗纹$\Delta L=(k+\frac{1}{2})\lambda$,条纹宽度$\Delta x=\frac{\lambda}{2\cos i}$;正入射时$\Delta L\approx2nh$,\textbf{相邻条纹对应厚度差}$\Delta h=\frac{\lambda}{2n}$

\textbf{劈形薄膜等厚干涉条纹宽度}$\Delta x=\frac{\lambda}{2\alpha}$,其中$\lambda$--膜内的光波长,$\alpha$--劈的顶角;$\delta(\Delta L)=-2nh\sin i
\delta i+2n\cos i\delta h$,实际直接观察时条纹向劈尖凸;通过按压法可判断高低

\textbf{牛顿环}半径$r_k^2=kR\lambda$,其中$R$--透镜曲率半径,$\lambda$--空气膜中的光波长,$R=\frac{r_{k+m}^2-r_k^2}{m\lambda}$

当\textbf{增透膜}为低膜,即$n_1<n<n_2$,且$n=\sqrt{n_1n_2}$,膜厚$(\frac{k}{2}+\frac{1}{4})\lambda$时,完全消反射,其中$n$--增透膜折射率,$n_1$--空气折射率,$n_2$--介质折射率;高反射膜,换成等厚度高膜,即$n_1<n<n_2$,多次介质高反射膜效果更佳

\textbf{薄膜等厚干涉光程差}$\Delta L=2nh\cos i$,第$k$级条纹$\Delta L=k\lambda\Longrightarrow\cos i_k=\frac{k\lambda}{2nh}$,故$\cos i_{k+1}-\cos i_{k}=\frac{\lambda}{2nh}$,又$\cos i_{k+1}-\cos i_{k}\approx(\frac{d\cos i}{di})_{i=i_k}=-\sin i_k(i_{k+1}-i_{k})$,从而倾角较小时$r_{k+1}-r_{k}\propto i_{k+1}-i_{k}=\frac{-\lambda}{2nh\sin i_k}$;$i_k$越大,$h$越大,则$|\Delta r|$越小,条纹越密;$h$增大,则环形条纹外扩;使用拓展光源,衬度不受影响

拓展光源导致非定域干涉问题,在定域中心层衬度最大,其附近有干涉条纹,但由于瞳孔的的限制,较大的拓展光源并不妨碍观察到图像的衬度

\textbf{迈克尔逊干涉仪移过条纹数目与反射镜移动距离关系}$l=N\frac{\lambda}{2}$

\textbf{光源单色性对迈氏干涉衬度影响}$I(\Delta L)=I_0[1+\cos\delta]=I_0[1+\cos k\Delta L]$,其中$k=\frac{2\pi}{\lambda}$,\textbf{等强双线}$I(\Delta L)=I_1(\Delta L)+I_2(\Delta L)=2I_0[1+\cos(\frac{\Delta k}{2}\Delta L)\cos(k\Delta L)]$,其中$\Delta k=k_1-k_2\ll k=\frac{1}{2}(k_1+k_2)$,故衬比度$\gamma=|\cos(\frac{\Delta k}{2}\Delta L)|$,从最强到最弱$\Delta L=N_1\lambda_1=N_2\lambda_2=(N-\frac{1}{2})\lambda_2\Longrightarrow N_1=\frac{\lambda_2}{2(\lambda_2-\lambda_1)}=\frac{\lambda}{2\Delta\lambda}$,\textbf{衬度变化空间频率}$\frac{1}{2N_1\lambda_1}=\frac{\lambda_2-\lambda_1}{\lambda_1\lambda_2}\approx\frac{\Delta\lambda}{\lambda^2}(=\frac{\Delta k}{2\pi})$;谱密度积分得总光强$I_0\int_0^\infty i(\lambda)d\lambda=\frac{1}{\pi}\int_0^\infty i(k)dk$,\textbf{单色线宽}$I(\Delta L)=\frac{1}{\pi}\int_0^\infty i(k)[1+\cos k\Delta L]dk=I_0+\frac{1}{\pi}\int_0^\infty i(k)\cos(k\Delta L)dk$,若$i(k)$仅在$k_0\pm\Delta k/2$内等于常数$\pi I_0/\Delta k$,则$I(\Delta L)=I_0[1+\frac{\sin(\Delta k\Delta L/2)}{\Delta k\Delta L/2}\cos(k_0\Delta L)]$,故衬度$\gamma=|\frac{\sin(\Delta k\Delta L/2)}{\Delta k\Delta L/2}|$,超过最大光程差$\Delta L_{max}=\frac{2\pi}{\Delta k}=\frac{\lambda^2}{|\Delta\lambda|}$,条纹不可见

$l_0=v\tau_0=\frac{c}{n}\tau_0$或$L_0=c\tau_0$,其中$L_0$--\textbf{相干长度},$\tau_0$--\textbf{相干时间};若$a(k)$仅在$k_0\pm\Delta k/2$内等于常数$\pi\widetilde{A}/\Delta k$,则波列长度$\Delta L\approx\frac{\lambda^2}{\Delta\Lambda},\Delta\nu=-c\frac{\Delta\lambda}{\lambda^2}\Longrightarrow\bm{\tau_0\Delta\nu\approx1}$

\textbf{法布里-珀罗干涉仪}斯托克斯关系$r=-r',r^2+tt'=1$,$\widetilde{U}_1=-Ar',\widetilde{U}_2=Atr't'e^{i\delta},\widetilde{U}_n=Atr'^{2n-3}t'e^{(n-1)i\delta},\widetilde{U}_1'=Att',\widetilde{U}_n'=Atr'^{2n-2}t'e^{(n-1)i\delta}\Longrightarrow \widetilde{U}_T=\frac{Att'}{1-r^2e^{i\delta}}\Longrightarrow I_T=\widetilde{U}_T^*\widetilde{U}_T=\frac{I_0(1-r^2)^2}{1-2r^2\cos\delta+r^4}=\frac{I_0}{1+\frac{4R\sin^2(\delta/2)}{(1-R)^2}}$,其中$\delta=\frac{2\pi}{\lambda}\Delta L=\frac{4\pi nh\cos i}{\lambda},R=r^2$,$I_R=I_0-I_T=\frac{I_0}{1+\frac{(1-R)^2}{4R\sin^2(\delta/2)}}$;当$R\ll1$,$I_T=I_0[1-2R(1-\cos\delta)],I_R=2RI_0[1-\cos\delta]$;半峰宽$\varepsilon\triangleq\Delta\delta=\frac{2(1-R)}{\sqrt{R}}$,根据$\delta=4\pi nh\cos i/\lambda$,若单色光固定$\lambda$,则\textbf{角宽度}$|\Delta i_k|=\frac{\lambda\varepsilon}{4\pi nh\sin i_k}=\frac{\lambda}{4\pi nh\sin i_k}\frac{2(1-R)}{\sqrt{R}}$,腔长$h$越大,条纹越细锐,若非单色光固定$i=0$,则仅特定波长$\lambda_k$附近的光出现极大,每条谱线称一纵模,
\textbf{纵模间隔}$\nu_k=\frac{c}{\lambda_k}=\frac{kc}{2nh}\Longrightarrow\Delta\nu=\frac{c}{2nh}$,\textbf{单模线宽}$\varepsilon=d\delta=-4\pi nh\cos id\lambda/\lambda^2\Longrightarrow\Delta\lambda_k=\frac{\lambda^2\varepsilon}{4\pi nh\cos i}=\frac{\lambda}{\pi k}\frac{1-R}{\sqrt{R}}\Longrightarrow\Delta\nu_k=\frac{c\Delta\lambda}{\lambda^2}=\frac{c}{\pi k\lambda}\frac{1-R}{\sqrt{R}}$
F-P干涉仪色分辨本领$2nh\cos i_k=k\lambda\Longrightarrow\delta i_k=\frac{k}{2nh\sin}\delta\lambda$,最小波长间隔$\delta\lambda=\frac{\lambda}{\pi k}\frac{1-R}{\sqrt{R}}$,\textbf{色分辨本领}$\frac{\lambda}{\delta\lambda}=\pi k\frac{\sqrt{R}}{1-R}$
\end{document}
